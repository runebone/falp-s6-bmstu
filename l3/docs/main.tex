\documentclass[a4paper,oneside,14pt]{extarticle}

\include{preamble}

\begin{document}

\include{title}
\setcounter{page}{2}
\renewcommand{\contentsname}{СОДЕРЖАНИЕ}
% \tableofcontents

% -------------------------------------------------------------------

% \newpage
% \section{Практические задания}

% \newpage
% \section{Теоретические вопросы}

\begin{lstlisting}
;; 1. Напишите функцию, которая принимает целое число и возвращает первое четное число, не меньшее аргумента
(defun f1 (x)
  (cond ((evenp x) x)
        (t (+ 1 x))))
\end{lstlisting}

\begin{lstlisting}
;; 2. Напишите функцию, которая принимает число и возвращает число того же знака, но с модулем на 1 больше модуля аргумента
(defun f2 (x)
  (cond ((< x 0) (- (+ 1 (- x))))
        (t (+ 1 x))))
\end{lstlisting}

\begin{lstlisting}
;; 3. Напишите функцию, которая принимает два числа и возвращает список из этих чисел, расположенный по возрастанию
(defun f3 (x y)
  (cond ((> x y) (cons y (cons x nil)))
        (t (cons x (cons y nil)))))
\end{lstlisting}

\begin{lstlisting}
;; 4. Напишите функцию, которая принимает три числа и возвращает T только тогда, когда первое число расположено между вторым и третьим
(defun f4 (x y z)
  (cond ((and (< x y) (< y z)) t)
        (t nil)))
\end{lstlisting}

\begin{lstlisting}
;; 5. Каков результат вычисления следующих выражений
(and 'fee 'fie 'foe) ;; FOE
(or nil 'fie 'foe) ;; FIE
(and (equal 'abc 'abc) 'yes) ;; YES
(or 'fee 'fie 'foe) ;; FEE
(and nil 'fie 'foe) ;; NIL
(or (equal 'abc 'abc) 'yes) ;; T
\end{lstlisting}

\newpage

\begin{lstlisting}
;; 6. Написать предикат, который принимает два числа-аргумента и возвращает T, если первое число не меньше второго
(defun f6 (x y)
  (cond ((>= x y) t)
        (t nil)))
\end{lstlisting}

\begin{lstlisting}
;; 7. Какой из следующих двух вариантов предиката ошибочен и почему?
(defun pred1 (x)
  (and (numberp x) (plusp x))) ;; OK

(defun pred2 (x)
  (and (plusp x) (numberp x)))
;; Ошибка. Аргументы and вычисляются в порядке слева направо - сначала будет вычисляться (plusp x), который вернёт ошибку в случае, если x не число.
\end{lstlisting}

\begin{lstlisting}
;; 8. Решить задачу 4, используя для её решения инструкции:
;; - только if
;; - только cond
;; - только and/or
(defun f8if (x y z)
  (if (<= y x) nil
    (if (>= y z) nil t)))

(defun f8cond (x y z)
  (cond ((<= y x) nil)
        ((>= y z) nil)
        (t t)))

(defun f8andor (x y z)
  (and (> y x) (< y z)))
\end{lstlisting}

\newpage

\begin{lstlisting}
;; 9. Переписать функцию how-alike, приведённую в лекции и использующую cond, используя только конструкции if, and/or
(defun how_alike (x y)
  (cond ((or (= x y) (equal x y)) 'the_same)
        ((and (oddp x) (oddp y)) 'both_odd)
        ((and (evenp x) (evenp y)) 'both_even)
        (t 'different)))

(defun how_alike_if (x y)
  (if (= x y)
    'the_same
    (if (equal x y)
        'the_same
        (if (oddp x)
            (if (oddp y)
                'both_odd
                'different)
            (if (evenp y)
                'both_even
                'different)))))

(defun how_alike_andor (x y)
  (or (and (or (= x y) (equal x y)) 'the_same)
      (and (and (oddp x) (oddp y)) 'both_odd)
      (and (and (evenp x) (evenp y)) 'both_even)
      'different))
\end{lstlisting}

% -------------------------------------------------------------------

\end{document}
