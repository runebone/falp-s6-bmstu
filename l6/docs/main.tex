\documentclass[a4paper,oneside,14pt]{extarticle}

\include{preamble}

\begin{document}

\include{title}
\setcounter{page}{2}
\renewcommand{\contentsname}{СОДЕРЖАНИЕ}
% \tableofcontents

% -------------------------------------------------------------------

\begin{lstlisting}
; 1. Написать хвостовую рекурсивную функцию my-reverse, которая развернет верхний уровень своего списка-аргумента lst
(defun my-reverse* (lst res)
  (cond ((null lst) res)
        (t (my-reverse* (cdr lst) (cons (car lst) res)))))

(defun my-reverse (lst)
  (my-reverse* lst nil))
\end{lstlisting}

\begin{lstlisting}
; 3. Написать функцию, которая возвращает первый элемент списка-аргумента, который сам является непустым списком
(defun f3 (lst)
  (cond ((null lst) res)
        ((consp (car lst)) (car lst))
        (t (f3 (cdr lst)))))
\end{lstlisting}

\begin{lstlisting}
; 4. Написать функцию, которая выбирает из заданного списка только те числа, которые больше 1 и меньше 10 (Вариант: между двумя заданными границами)
(defun f4* (lst from to res)
  (cond ((null lst) res)
        ((and (numberp (car lst))
              (> (car lst) from) 
              (< (car lst) to))
         (f4* (cdr lst) from to (cons (car lst) res)))
        (t (f4* (cdr lst) from to res))))

(defun f4 (lst)
  (my-reverse (f4* lst 1 10 nil)))
\end{lstlisting}

\begin{lstlisting}
; 5. Напишите рекурсивную функцию, которая умножает на заданное число-аргумент все числа из заданного списка-аргумента, когда
; a) все элемента списка - числа,
; б) элементы списка - любые объекты.
(defun all-numbersp (lst)
  (cond ((null lst) nil)
        ((and (numberp (car lst)) (null (cdr lst))) t)
        ((not (numberp (car lst))) nil)
        (t (all-numbersp (cdr lst)))))

(defun f5a (lst n)
  (cond ((all-numbersp lst) (mapcar #'(lambda (x) (* x n)) lst))
        (t lst)))

(defun apply-if-number (a fn)
  (cond ((numberp a) (apply fn (cons a nil)))
        (t a)))

(defun self (x) x)

(defun f5b (lst n)
  (mapcar #'(lambda (x)
              (apply-if-number x #'(lambda (y) (* y n)))) lst))
\end{lstlisting}

\begin{lstlisting}
; 8. Написать рекурсивную версию (с именем rec-add) вычисления суммы чисел заданного списка:
; а) одноуровневого смешанного,
; б) структурированного.
(defun rec-add-a* (lst sum)
  (cond ((null lst) sum)
        ((numberp (car lst)) (rec-add-a* (cdr lst)
                                         (+ sum (car lst))))
        (t (rec-add-a* (cdr lst) sum))))

(defun rec-add-a (lst)
  (rec-add-a* lst 0))

(defun rec-add-b* (lst sum)
  (cond ((null lst) sum)
        ((numberp (car lst)) (rec-add-b* (cdr lst)
                                         (+ sum (car lst))))
        ((consp (car lst))
         (+ sum (rec-add-b* (car lst) 0) (rec-add-b* (cdr lst) 0)))
        (t (rec-add-b* (cdr lst) sum)))) ; meeeh

(defun rec-add-b (lst)
  (rec-add-b* lst 0))
\end{lstlisting}

\begin{lstlisting}
; 9. Написать рекурсивную версию с именем recnth функции nth.
(defun recnth (n lst)
  (cond ((and (numberp n) (= n 0)) (car lst))
        ((and (numberp n) (> n 0)) (recnth (- n 1) (cdr lst)))
        (t nil)))
\end{lstlisting}

\begin{lstlisting}
; 10. Написать рекурсивную функцию allodd, которая возвращает t, когда все элементы списка нечетные.
(defun allodd (lst)
  (cond ((not (numberp (car lst))) nil)
        ((and (oddp (car lst)) (null (cdr lst))) t)
        ((oddp (car lst)) (allodd (cdr lst)))
        (t nil)))
\end{lstlisting}

\begin{lstlisting}
; 11. Написать рекурсивную функцию, которая возвращает первое нечетное число из списка (структурированного), возможно создавая некоторые вспомогательные функции.
(defun first-odd (lst)
  (cond ((and (numberp lst) (oddp lst)) lst)
        ((atom lst) nil)
        (t (or (first-odd (car lst))
               (first-odd (cdr lst))))))
\end{lstlisting}

\begin{lstlisting}
; 12. Используя cons-дополняемую рекурсию с одним тестом завершения, написать функцию, которая получает как аргумент список чисел, а возвращает список квадратов этих чисел в том же порядке.
(defun f12* (lst reslst)
  (cond ((null lst) reslst)
        (t (f12* (cdr lst) (cons (* (car lst) (car lst)) reslst)))))

(defun f12 (lst)
  (my-reverse (f12* lst nil)))
\end{lstlisting}

% -------------------------------------------------------------------

\end{document}
