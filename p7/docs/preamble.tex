\usepackage{cmap} % Улучшенный поиск русских слов в полученном pdf-файле
\usepackage[T2A]{fontenc} % Поддержка русских букв
\usepackage[utf8]{inputenc} % Кодировка utf8
\usepackage[english,russian]{babel} % Языки: русский, английский

\usepackage[14pt]{extsizes}

\usepackage{graphicx}
\usepackage{multirow}

\usepackage{tikz}
\usetikzlibrary{shapes, shapes.geometric, arrows, arrows.meta, positioning}

\usepackage{caption}
\captionsetup{labelsep=endash}
\captionsetup[figure]{name={Рисунок}}

\usepackage{amsmath}
\usepackage{amsfonts}

\usepackage{geometry}
\geometry{left=30mm}
\geometry{right=10mm}
\geometry{top=20mm}
\geometry{bottom=20mm}

\usepackage{enumitem}

\usepackage{tabularx}
\usepackage{longtable}
\usepackage{adjustbox}
\usepackage{threeparttable}

% Переопределение стандартных \section, \subsection, \subsubsection по ГОСТу;
\usepackage{titlesec}[explicit]
\titleformat{name=\section,numberless}[block]{\normalfont\large\bfseries\centering}{}{0pt}{}
\titleformat{\section}[block]{\normalfont\large\bfseries}{\thesection}{1em}{}
\titlespacing\section{\parindent}{*4}{*4}

\titleformat{\subsection}[hang]
{\bfseries\large}{\thesubsection}{1em}{}
\titlespacing\subsection{\parindent}{*2}{*2}

\titleformat{\subsubsection}[hang]
{\bfseries\large}{\thesubsubsection}{1em}{}
\titlespacing\subsubsection{\parindent}{*2}{*2}

\usepackage{url}

% Переопределение их отступов до и после для 1.5 интервала во всем документе
\usepackage{setspace}
\onehalfspacing % Полуторный интервал
\frenchspacing
\setlength\parindent{1.25cm}

\usepackage{indentfirst} % Красная строка

% Настройки оглавления
\usepackage{xcolor}
\usepackage{multirow}

% Гиперссылки
\usepackage[pdftex]{hyperref}
\hypersetup{hidelinks}

% Дополнительное окружения для подписей
\usepackage{array}
\newenvironment{signstabular}[1][1]{
	\renewcommand*{\arraystretch}{#1}
	\tabular
}{
	\endtabular
}

\usepackage{enumitem} 
\setenumerate[0]{label=\arabic*)} % Изменение вида нумерации списков
\renewcommand{\labelitemi}{---}

% Листинги 
\usepackage{courier}
\usepackage{listings}
\usepackage{chngcntr} % Listings counter within section set main.tex after begin document
\usepackage{float} % Place figures anywhere you want; ignore floating
\floatstyle{plaintop}
\newfloat{code}{H}{myc}

% Для листинга кода:
\lstset{
	basicstyle=\small\ttfamily,
	language=prolog,
	numbers=left,
	numbersep=5pt,
	% stepnumber=1,
	xleftmargin=17pt,
	% showstringspaces=false,
	numbersep=5pt,
	frame=single,
	tabsize=4,
	captionpos=b,
	breaklines=true,
	breakatwhitespace=true,
	escapeinside={\#*}{*)},
	inputencoding=utf8x,
	backgroundcolor=\color{white},
	% numberstyle=,%\tiny
	keywordstyle=\color{blue},
	stringstyle=\color{red!90!black},
	commentstyle=\color{green!50!black}
}
\lstset{
 morekeywords={domains, predicates, clauses, goal}
}

\lstset{
	literate=
	{а}{{\selectfont\char224}}1
	{б}{{\selectfont\char225}}1
	{в}{{\selectfont\char226}}1
	{г}{{\selectfont\char227}}1
	{д}{{\selectfont\char228}}1
	{е}{{\selectfont\char229}}1
	{ё}{{\"e}}1
	{ж}{{\selectfont\char230}}1
	{з}{{\selectfont\char231}}1
	{и}{{\selectfont\char232}}1
	{й}{{\selectfont\char233}}1
	{к}{{\selectfont\char234}}1
	{л}{{\selectfont\char235}}1
	{м}{{\selectfont\char236}}1
	{н}{{\selectfont\char237}}1
	{о}{{\selectfont\char238}}1
	{п}{{\selectfont\char239}}1
	{р}{{\selectfont\char240}}1
	{с}{{\selectfont\char241}}1
	{т}{{\selectfont\char242}}1
	{у}{{\selectfont\char243}}1
	{ф}{{\selectfont\char244}}1
	{х}{{\selectfont\char245}}1
	{ц}{{\selectfont\char246}}1
	{ч}{{\selectfont\char247}}1
	{ш}{{\selectfont\char248}}1
	{щ}{{\selectfont\char249}}1
	{ъ}{{\selectfont\char250}}1
	{ы}{{\selectfont\char251}}1
	{ь}{{\selectfont\char252}}1
	{э}{{\selectfont\char253}}1
	{ю}{{\selectfont\char254}}1
	{я}{{\selectfont\char255}}1
	{А}{{\selectfont\char192}}1
	{Б}{{\selectfont\char193}}1
	{В}{{\selectfont\char194}}1
	{Г}{{\selectfont\char195}}1
	{Д}{{\selectfont\char196}}1
	{Е}{{\selectfont\char197}}1
	{Ё}{{\"E}}1
	{Ж}{{\selectfont\char198}}1
	{З}{{\selectfont\char199}}1
	{И}{{\selectfont\char200}}1
	{Й}{{\selectfont\char201}}1
	{К}{{\selectfont\char202}}1
	{Л}{{\selectfont\char203}}1
	{М}{{\selectfont\char204}}1
	{Н}{{\selectfont\char205}}1
	{О}{{\selectfont\char206}}1
	{П}{{\selectfont\char207}}1
	{Р}{{\selectfont\char208}}1
	{С}{{\selectfont\char209}}1
	{Т}{{\selectfont\char210}}1
	{У}{{\selectfont\char211}}1
	{Ф}{{\selectfont\char212}}1
	{Х}{{\selectfont\char213}}1
	{Ц}{{\selectfont\char214}}1
	{Ч}{{\selectfont\char215}}1
	{Ш}{{\selectfont\char216}}1
	{Щ}{{\selectfont\char217}}1
	{Ъ}{{\selectfont\char218}}1
	{Ы}{{\selectfont\char219}}1
	{Ь}{{\selectfont\char220}}1
	{Э}{{\selectfont\char221}}1
	{Ю}{{\selectfont\char222}}1
	{Я}{{\selectfont\char223}}1
}

% Работа с изображениями и таблицами; переопределение названий по ГОСТу
\usepackage{caption}
\captionsetup[figure]{name={Рисунок},labelsep=endash}
\captionsetup[table]{singlelinecheck=false, labelsep=endash}

\usepackage[justification=centering]{caption} % Настройка подписей float объектов	

\usepackage{csvsimple}

\usepackage{ulem} % Нормальное нижнее подчеркивание
\usepackage{hhline} % Двойная горизонтальная линия в таблицах
\usepackage[figure,table]{totalcount} % Подсчет изображений, таблиц
\usepackage{rotating} % Поворот изображения вместе с названием
\usepackage{lastpage} % Для подсчета числа страниц

\makeatletter
\renewcommand\@biblabel[1]{#1.} % [1] -> 1. in bibliography
\makeatother

\usepackage{ragged2e} % Перенос слов на следующую строку
\usepackage{pdfpages}

\usepackage{blindtext}

% \usepackage[
%     backend=biber,
% 	style=gost-numeric,
% 	% style=numeric-comp,
% 	language=auto,
% 	autolang=other,
% 	sorting=none
% ]{biblatex}
% \addbibresource{bibliography.bib}
% \usepackage{xparse} % \NewDocumentCommand for creating custom commands
% \NewDocumentCommand{\printbib}{m}
% {\printbibliography[title={#1}]\addcontentsline{toc}{section}{#1}}
